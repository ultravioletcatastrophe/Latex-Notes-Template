\documentclass[justified, nofonts, notitlepage, openany]{tufte-book}

%%% Sets numbering depth to section level (e.g, no numbered subsections)
\setcounter{secnumdepth}{1}

\usepackage{notes}
\geometry{left=1in, right=3in,top = 1in, bottom=1in}

%%% Removes paragraph indentation and changes paragraph line skip
\makeatletter
\renewcommand{\@tufte@reset@par}{%
  \setlength{\RaggedRightParindent}{0pc}%
  \setlength{\JustifyingParindent}{0pc}%
  \setlength{\parindent}{0pc}%
  \setlength{\parskip}{12pt}%
}
\@tufte@reset@par
\makeatother


\begin{document}
%%% The Title and Author only need to be set once at the start of the document. If you take notes for multiple courses in the same document (for example, in a multi-semester sequence for the same course), you can separate the courses with a new Part, and the semester, lecturer, and course only need to be set once at the start of the new course.
\title{Math 254: Algebraic Number Theory}
\author{Abhishek Shivkumar}

\part{Non-Riemannian Hypersquares}
\semester{Fall 2020}
\lecturer{Alexander Grothendieck}
\course{Math 256C: From Schemes to Machinations}

\chapter{Lecture 1: 26 August}
\section{Tutorial}

Here is where lecture content goes, generally a summary or transcription of what is being said or written. Here is a theorem:
\begin{theorem}[Kontsevich]
    The number $N_d$ of rational plane curves of degree $d$ passing through $3d-1$ points in general position is given recursively by $$N_d = \sum_{d_A + d_B = d} N_{d_A} N_{d_B} d_A^2 d_B\left(d_B\binom{3d-4}{3d_A -2} - d_A \binom{3d-4}{3d_A-1}\right)$$
\end{theorem}
\marginnote{Here is a margin note: note that the above result was only obtained in the early 1990s, using ideas from theoretical physics.}

The above result, is, of course, thoroughly unrelated to the following fact:
\begin{lemma}
    In a $k$-free graph on $n$ vertices, there are at most $\binom{k-1}{r} (\frac{n}{k-1})^r$ $r$-cliques.
\end{lemma}
Setting $r=2$ in the above, we recover the following result:
\begin{corollary}[Turan's Theorem][hereyouputthetagforreferences]
    In a $k$-free graph on $n$ vertices, there are at most $\frac{k-2}{k-1} \frac{n^2}{2}$ edges. 
\end{corollary}
\marginnote{Note that this upper bound can be mildly strengthened into a strict upper bound by considering the different cases for $n$ modulo $k-1$. In particular, if $r$ is the remainder when $n$ is divided by $k-1$, then the upper bound on edges is $$\frac{k-2}{k-1}\frac{n^2 - r^2}{2} + \binom{r}{2}$$ and $k$-free graphs with precisely that many edges can be straightforwardly constructed.}
You can reference any theorem box if you add a reference tag (see the $\LaTeX$ code at Corollary~\ref{cor:hereyouputthetagforreferences} for formatting, and see the style file for the reference prefixes for each theorem style).

Unnnumbered versions of all the theorem boxes exist:
\begin{proposition*}[Hurwitz][hurwitz]
    The group of orientation-preserving conformal automorphisms of a compact Riemann surface of genus $g > 1$ has order at most $84(g-1)$.
\end{proposition*}
\marginnote{References do not work for unnumbered theorems.}
\begin{proof}
    There is also a proof environment.
\end{proof}


\end{document}