\documentclass[justified, nofonts, notitlepage, openany, debug, nobib]{tufte-book}

%%% Sets numbering depth to section level (e.g, no numbered subsections)
\setcounter{secnumdepth}{1}

\usepackage{notes}
\addbibresource{references.bib}
\geometry{left=1in, right=3in,top = 1in, bottom=1in}

%%% Removes paragraph indentation and changes paragraph line skip
\makeatletter
\renewcommand{\@tufte@reset@par}{%
  \setlength{\RaggedRightParindent}{0pc}%
  \setlength{\JustifyingParindent}{0pc}%
  \setlength{\parindent}{0pc}%
  \setlength{\parskip}{12pt}%
}
\@tufte@reset@par
\makeatother

% \talkmode %if you only need one chapter, use \talkmode which fixes theorem numberings (omits chapter #) and uses the chapter title instead of the title itself in headers (for talk notes I tend to use \title for the name of the seminar I gave the talk in)

\begin{document}
%%% The Title and Author only need to be set once at the start of the document. If you take notes for multiple courses in the same document (for example, in a multi-semester sequence for the same course), you can separate the courses with a new Part, and the semester, lecturer, and course only need to be set once at the start of the new course.
\title{Math 256C: From Schemes to Conspiracies}
\author{Abhishek Shivkumar}

% \part{Non-Riemannian Hypersquares}
\semester{Fall 2020}
\lecturer{Professor Alexander Grothendieck}

\chapter{Chapter I: Points with Endomorphisms}
\section{Tutorial}
\subsection{Basics}

\bmn{Lecture 1: September 3$^\text{rd}$, 1752}
I tend to use chapter headings for larger sections of a course (as opposed to individual lectures); to keep track of where lectures start you can use the \texttt{bmn} (boxed margin note) command, which writes to the margin.

Here is where lecture content goes, generally a summary or transcription of what is being said or written. Here is a theorem:
\begin{theorem}[Kontsevich][][kontsevich-manin]
    The number $N_d$ of rational plane curves of degree $d$ passing through $3d-1$ points in general position is given recursively by $$N_d = \sum_{d_A + d_B = d} N_{d_A} N_{d_B} d_A^2 d_B\left(d_B\binom{3d-4}{3d_A -2} - d_A \binom{3d-4}{3d_A-1}\right)$$
\end{theorem}
\marginnote{Here is a margin note: I use these generally to annotate my own thoughts or questions during lecture.

You can have multi-paragraph margin notes, which are configured to not have indented paragraphs.}

The above result, is, of course, thoroughly unrelated to the following fact:
\begin{lemma}
    In a $k$-free graph on $n$ vertices, there are at most $\binom{k-1}{r} (\frac{n}{k-1})^r$ $r$-cliques.
\end{lemma}
Setting $r=2$ in the above, we recover the following result:\sidenote{There are also numbered sidenotes.}
\begin{corollary}[Turan's Theorem][turan]
    In a $k$-free graph on $n$ vertices, there are at most $\frac{k-2}{k-1} \frac{n^2}{2}$ edges. 
\end{corollary}
You can insert a hyperlinked reference for any theorem box if you add a reference tag (see formatting below), e.g, \verb_Corollary_$\sim$\verb_\ref{cor:turan}_ becomes Corollary~\ref{cor:turan} (see the style file for the reference prefixes for each theorem style).
\begin{proof}
    There is also a proof environment; the proof heading is configured to live in the left margin.
\end{proof}

Citations live in the right margin\cite[232]{fantechi}. Repeated citations appear as Ibid\cite{fantechi}. The available theorem boxes are \texttt{theorem}, \texttt{lemma}, \texttt{corollary}, \texttt{proposition}, \texttt{definition}, \texttt{example}, \texttt{remark}, \texttt{question}, \texttt{exercise}, \texttt{counterexample}, and \texttt{conjecture}. 

Unnnumbered versions of all the theorem boxes exist:

\begin{proposition*}[Hurwitz][fantechi,hartshorne]
    The group of orientation-preserving conformal automorphisms of a compact Riemann surface of genus $g > 1$ has order at most $84(g-1)$.
\end{proposition*}
\marginnote{For obvious reasons, internal references do not work for unnumbered theorems.}

Formatting for a theorem box is as follows:
\begin{verbatim}
\begin{theorem}[<theorem name/author>]%
    [<optional tag for references>]%
    [<optional citation keys/tags (comma separated) for this theorem>]

    ... <theorem statement goes here> ...
\end{theorem}
\end{verbatim}
The above is the \textit{only} way to use citations inside one of the supplied theorem environments due to some incompatbilities between the \texttt{tcolorbox} package and the \texttt{tufte-book} class. Numbered sidenotes do not work inside theorem boxes, unnumbered marginnotes often work fine.


%Unfortunately, citations also do not work inside the tcolorboxes, see https://github.com/T-F-S/tcolorbox/issues/192

% \subsection{Advanced}
% \marginnote{Theorem box numbering is not affected by subsection; this is set by \texttt{secnumdepth} at the top of the preamble.}
% Using \texttt{https://q.uiver.app/}, we can curve arrows more flexibly in commutative diagrams than \texttt{tikz-cd} normally allows:
% \[\begin{tikzcd}
%     & {\PP^n \setminus \{Q\}} & {W_0} \\
%     X & {\PP^n \setminus V} & {W_0\setminus V'} & W
%     \arrow["{\varphi_0}", from=1-2, to=1-3]
%     \arrow[from=2-2, to=1-2]
%     \arrow["{\varphi_0|_{\PP^n \setminus V}}", from=2-2, to=2-3]
%     \arrow["{\varphi'}", from=2-3, to=2-4]
%     \arrow[hook, from=2-1, to=1-2]
%     \arrow[hook, from=2-1, to=2-2]
%     \arrow[hook, from=2-3, to=1-3]
%     \arrow["\varphi"', curve={height=12pt}, from=2-1, to=2-4]
% \end{tikzcd}\]

% Finally, there is (in addition to the default environments provided by \texttt{tufte-book}) a two-column environment, which I mostly use to spam examples after a big theorem:
% \begin{twocol}
% \begin{example}
%     In $\CC \PP^2$ with its canonical orientation as a complex manifold, we have $H^2 = H_2 = \ZZ$. Lines $L$ in $\CC \PP^2$ are embedded copies of $\CC \PP^1 = S^2$, with a single fundamental class $[L] \in H_2$ representing any such line. To compute the self-intersection $[L] \cdot [L]$, choose two representatives $L$ and $L'$ which intersect transversely (in a point, by basic intersection theory); since $\CC \PP^1$ is canonically oriented as well, and the orientations of $L$ and $L'$ agree with that of $\CC \PP^2$ (since they are subspaces of $\CC \PP^2$), we have that $[L] \cdot [L] = 1$. 
% \end{example}
% \begin{example}
%     In $S^2 \times S^2$, the second homology group is $\ZZ^2$, generated by $h = [S^2 \times \{x\}]$ and $v = [\{x\} \times S^2]$. $S^2 \times \{x\}$ and $\{x\} \times S^2$ intersect transversely at $(x,x)$, and, regarding $S^2$ with its canonical orientation as a complex manifold ($\CC \PP^1$), the orientations agree to give $h \cdot v = 1$. To compute $h \cdot h$, intersect representatives $S^2 \times \{x\}$ and $S^2 \times \{y\}$ for $x \neq y$, from which it is set-theoretically clear that $h^2 = 0$ (and similarly for $v$). Thus, the intersection form as a matrix in the basis $(h,v)$ can be written as $\begin{psmallmatrix} 0 & 1 \\ 1 & 0 \end{psmallmatrix}$ This intersection form is therefore \textit{even}, meaning that for any $k = ah+bv$, $k \cdot k = 2ab$ is even.
% \end{example}
% \end{twocol}

%%% Removed as use of this environment can often cause the insertion of unwanted paragraph breaks, see https://tex.stackexchange.com/questions/695327/tcolorbox-multicols-tufte-book-inserting-unwanted-paragraph-breaks, https://github.com/T-F-S/tcolorbox/issues/244, https://github.com/T-F-S/tcolorbox/issues/245. Seems unfixable.

\nocite{*} %Print all references, even those never cited in the body
\printbibliography[title=References]
\end{document}