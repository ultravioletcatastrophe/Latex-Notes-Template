\documentclass[justified, nofonts, notitlepage, openany]{tufte-book}

%%% Sets numbering depth to section level (e.g, no numbered subsections)
\setcounter{secnumdepth}{1}

\usepackage{notes}
\geometry{left=1in, right=3in,top = 1in, bottom=1in}

%%% Removes paragraph indentation and changes paragraph line skip
\makeatletter
\renewcommand{\@tufte@reset@par}{%
  \setlength{\RaggedRightParindent}{0pc}%
  \setlength{\JustifyingParindent}{0pc}%
  \setlength{\parindent}{0pc}%
  \setlength{\parskip}{12pt}%
}
\@tufte@reset@par
\makeatother


\begin{document}
%%% The Title and Author only need to be set once at the start of the document. If you take notes for multiple courses in the same document (for example, in a multi-semester sequence for the same course), you can separate the courses with a new Part, and the semester, lecturer, and course only need to be set once at the start of the new course.
\title{Math 254: Algebraic Number Theory}
\author{Abhishek Shivkumar}

\part{Non-Riemannian Hypersquares}
\semester{Fall 2020}
\lecturer{Alexander Grothendieck}
\course{Math 256C: From Schemes to Machinations}

\chapter{Lecture 1: 26 August}
\section{Tutorial}

Here is where lecture content goes, generally a summary or transcription of what is being said or written. Here is a theorem:
\begin{theorem}[Kontsevich]
    The number $N_d$ of rational plane curves of degree $d$ passing through $3d-1$ points in general position is given recursively by $$N_d = \sum_{d_A + d_B = d} N_{d_A} N_{d_B} d_A^2 d_B\left(d_B\binom{3d-4}{3d_A -2} - d_A \binom{3d-4}{3d_A-1}\right)$$
\end{theorem}
\marginnote{Here is a margin note: note that the above result was only obtained in the early 1990s, using ideas from theoretical physics.}

The above result, is, of course, thoroughly unrelated to the following fact:
\begin{lemma}
    In a $k$-free graph on $n$ vertices, there are at most $\binom{k-1}{r} (\frac{n}{k-1})^r$ $r$-cliques.
\end{lemma}
Setting $r=2$ in the above, we recover the following result:
\begin{corollary}[Turan's Theorem][hereyouputthetagforreferences]
    In a $k$-free graph on $n$ vertices, there are at most $\frac{k-2}{k-1} \frac{n^2}{2}$ edges. 
\end{corollary}
\marginnote{Note that this upper bound can be mildly strengthened into a strict upper bound by considering the different cases for $n$ modulo $k-1$. In particular, if $r$ is the remainder when $n$ is divided by $k-1$, then the upper bound on edges is $$\frac{k-2}{k-1}\frac{n^2 - r^2}{2} + \binom{r}{2}$$ and $k$-free graphs with precisely that many edges can be straightforwardly constructed.}
You can reference any theorem box if you add a reference tag (see the $\LaTeX$ code at Corollary~\ref{cor:hereyouputthetagforreferences} for formatting, and see the style file for the reference prefixes for each theorem style).

Unnnumbered versions of all the theorem boxes exist:
\begin{proposition*}[Hurwitz]
    The group of orientation-preserving conformal automorphisms of a compact Riemann surface of genus $g > 1$ has order at most $84(g-1)$.
\end{proposition*}
\marginnote{References do not work for unnumbered theorems.}
\begin{proof}
    There is also a proof environment.
\end{proof}

Some miscellaneous things:
\begin{center}
\begin{tikzpicture}[tqft/cobordism/.style={draw},
        tqft/view from=outgoing, tqft/boundary separation=30pt,
        tqft/cobordism height=40pt, tqft/circle x radius=8pt,
        tqft/circle y radius=3pt, tqft/every boundary component/.style={draw,rotate=90}]

        \pic[tqft/cylinder,rotate=90,name=a,anchor={(1,0)}];
        \pic[tqft,
            incoming boundary components=0,
            outgoing boundary components=2,
            rotate=90,name=b,anchor={(0,0)}];
        \pic[tqft,
            incoming boundary components=2,
            outgoing boundary components=0,
            rotate=90,name=c,at=(a-outgoing boundary)];
        \pic[tqft/cylinder,rotate=90,name=d,at=(b-outgoing boundary 2)];
        \node at ([xshift=-11pt,yshift = -65pt]b-between outgoing 1 and 2) {$(\varphi \sqcup \id_{\overline{E}}) \hspace{1pt}\circ$};
        \node at ([xshift=+12pt,yshift = -35pt]c-between incoming 1 and 2) {$(\id_{\overline{E}} \sqcup \hspace{1pt}\psi)$};
        \node[pin=20:$E$] at (c-incoming boundary 2) {};
        \node at ([xshift=-10pt]a-incoming boundary 1) {$\overline{E}$};
        \node at ([xshift=10pt]d-outgoing boundary) {$\overline{E}$};

        \path (b-outgoing boundary 1) ++(2,-.3) node[font=\LARGE] {\(=\)};

        \pic[tqft/cylinder,rotate=90,name=e,anchor={(.3,-3)}];
        \pic[tqft/cylinder,rotate=90,name=f,at=(e-outgoing boundary)];
        \pic[tqft/cylinder,rotate=90,name=g,at=(f-outgoing boundary)];
        \pic[tqft/cylinder,rotate=90,name=h,at=(g-outgoing boundary)];
        
        \node at ([yshift = -20pt]f-outgoing boundary) {$\id_{\overline{E}}$};

\end{tikzpicture} 
\end{center}

\marginnote{Associativity of the product arising from a 2D TQFT.}
\begin{center}
\begin{tikzpicture}[tqft/cobordism/.style={draw},
         tqft/view from=outgoing, tqft/boundary separation=40pt,
         tqft/cobordism height=40pt, tqft/circle x radius=8pt,
         tqft/circle y radius=3pt, tqft/every boundary component/.style={draw,rotate=90}]

         \pic[tqft/reverse pair of pants,draw,rotate=90,name=a];
         \pic[tqft/reverse pair of pants,draw,rotate=90,name=c,anchor=outgoing boundary, at=(a-incoming boundary 1)];
         \pic[tqft/cylinder to prior,draw,rotate=90,name=d,anchor=outgoing boundary, at=(a-incoming boundary 2)];

         \path (a-outgoing boundary) ++(1,0) node[font=\LARGE] {\(\cong\)};

         \pic[tqft/reverse pair of pants,draw,rotate=90,name=b,anchor=outgoing boundary, at=(a-outgoing boundary), yshift=-140pt];
         \pic[tqft/reverse pair of pants,draw,rotate=90,name=e,anchor=outgoing boundary, at=(b-incoming boundary 2)];
         \pic[tqft/cylinder to next,draw,rotate=90,name=f,anchor=outgoing boundary, at=(b-incoming boundary 1)];

         \node at ([xshift=-10pt]d-incoming boundary) {$a$};
         \node at ([xshift=-10pt,yshift=2pt]d-outgoing boundary) {$a$};
         \node at ([xshift=-10pt]c-incoming boundary 2) {$b$};
         \node at ([xshift=-10pt]c-incoming boundary 1) {$c$};

         \node[pin=-90:$b*c$] at (c-outgoing boundary) {};
         \node[pin=-90:$a*(b*c)$] at (a-outgoing boundary) {};         

         \node at ([xshift=-10pt]e-incoming boundary 2) {$a$};
         \node at ([xshift=-10pt]e-incoming boundary 1) {$b$};
         \node[pin=90:$a*b$] at (e-outgoing boundary) {};

         \node at ([xshift=-10pt]f-incoming boundary) {$c$};
         \node at ([xshift=-10pt,yshift=-2pt]f-outgoing boundary) {$c$};

         \node[pin=-90:$(a*b)*c$] at (b-outgoing boundary) {};         


\end{tikzpicture}
\end{center}
\marginnote{Diagrammatic arguments for two-dimensional TQFTs often rely on this strategy of placing vectors at boundary components and using the fact that morphisms in $\cob_n$ are only defined up to diffeomorphisms to prove identities in $Z(S^1)$.}

Using \texttt{https://q.uiver.app/} (whose style file is included in our style file for convenience), we can curve arrows more flexibly in commutative diagrams than tikz-cd normally allows:
\[\begin{tikzcd}
    & {\PP^n \setminus \{Q\}} & {W_0} \\
    X & {\PP^n \setminus V} & {W_0\setminus V'} & W
    \arrow["{\varphi_0}", from=1-2, to=1-3]
    \arrow[from=2-2, to=1-2]
    \arrow["{\varphi_0|_{\PP^n \setminus V}}", from=2-2, to=2-3]
    \arrow["{\varphi'}", from=2-3, to=2-4]
    \arrow[hook, from=2-1, to=1-2]
    \arrow[hook, from=2-1, to=2-2]
    \arrow[hook, from=2-3, to=1-3]
    \arrow["\varphi"', curve={height=12pt}, from=2-1, to=2-4]
\end{tikzcd}\]

\begin{question}
    Are these diagrams really necessary to include in a sample file?
\end{question}

\end{document}